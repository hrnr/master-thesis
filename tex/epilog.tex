\chapter*{Conclusion}
\addcontentsline{toc}{chapter}{Conclusion}

This work presented a novel map-merging algorithm for merging \gls{3D} point cloud maps in multi-robot systems. The algorithm is based on feature-matching transformation estimation and works solely on point cloud maps without any additional auxiliary information. This makes the algorithm applicable in heterogeneous multi-robot systems and the algorithm can work with different \gls{SLAM} approaches and sensor types. To the best of my knowledge the presented approach is the first map-merging algorithm working directly on point clouds without any extra information.

The work showed feasibility of feature-matching approach for registration of low-density point cloud maps produced by \gls{SLAM} algorithms while using \gls{3D} point cloud features typically employed with high-density sensor data.

A reciprocal descriptor matching algorithm was introduced for estimating the initial transformation using feature-matching. The algorithm requires very little parametrisation and exhibited good performance across descriptors in the evaluation. In the most configurations it outperforms \gls{SAC-IA} algorithm for initial alignment available in the \gls{PCL} library.

The map-merging algorithm has been evaluated on real-word datasets captured by both aerial and ground-based robots with variety of stereo rig cameras and active \gls{RGB-D} cameras. It has been evaluated in both indoor and outdoor environments ranging from forest to a single furnished room. The datasets used for evaluation include both well-established benchmark robotics datasets as well as my own experiments.

The proposed algorithm was implemented as a \gls{ROS} package. To the best of my knowledge it is the first \gls{ROS} package for map-merging of \gls{3D} maps. The package has been submitted to the \gls{ROS} distribution, the binary packages are distributed with the \gls{ROS} since the \gls{ROS} Melodic Morenia release. The implementation does not require any particular communication solution between robots and can work with \gls{ROS} in both multi-master and single-master setup. Likewise the implementation does not presume any particular \gls{SLAM} method, nor any particular sensor and uses a portable point cloud map representation, which make it compatible with existing readily available \gls{SLAM} implementations.

While the selected map representation enables great interoperability with existing software, the monolithic point clouds do not permit efficient repairing of mapping errors in the merged map. A pose graph of point clouds representation would be beneficial for the map-merging, but there is no standardised message format in the \gls{ROS} for such a representation nor there is a common graph representation established across different \gls{SLAM} implementations. In the future it would be beneficial to introduce a portable pose graph representation to the \gls{ROS}, as discussed in section~\ref{sec:map-representation}, support it within the core \gls{ROS} packages and promote its usage across \gls{SLAM} implementations. This representation would allow the presented algorithm to work on sub-maps in the pose graph and repair the mapping errors in the merged map.

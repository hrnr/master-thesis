\chapter*{Introduction}
\addcontentsline{toc}{chapter}{Introduction}

Multi-robot systems are established research area within robotics and artificial intelligence with growing number of applications. A multi-robot system, consisting of several individual robots, can improve efficiency in terms of both performance and robustness over a single robot. A team of robots can also solve tasks that cannot be tackled by a single robot. Multi-robot systems can scale to large environments and to the most demanding applications.

Multi-robot systems have been deployed to several application domains including space exploration~\citep{goldberg2002distributedspace,huntsberger2003campout}, autonomous patrolling~\citep{parker2003parolling100}, disaster rescue and victim search, aerial surveillance, unmanned delivery, mine cleaning, snow removal~\citep{choset2001coverage}, assembly of large-scale buildings or planetary habitat~\citep{goldberg2002distributedspace} and robotic football~\citep{asada1999robocup}. Those applications require robots to operate in  dynamic environments, with a high degree of uncertainty and external changes caused by other robots, humans and other agents that are not part of the multi-robot system itself.

To be able to solve such demanding problems multi-robot systems rely on distributed planning, communication and control algorithms. This work presents an algorithm for estimating positions of the robots in the shared environment and for building the global map of the environment (map-merging) without any previous knowledge or assumptions about the environment. Knowledge of positions of the robots and the map of the environment are crucial elements of effective collaborative planning and coordination. For most of the multi-robot systems this knowledge is essential and required for the operation, making the map-merging a key problem to solve. Multi-robot systems that don't have knowledge of the presence of other robots (\textit{unaware systems}~\citep{farinelli2004multirobot}) are normally used only for very simple tasks, as noted by~\citet{farinelli2004multirobot}.

A typical multi-robot system consists of many software parts typically structured in a multi-layer architecture. The implementation presented in this work leverages modularity of the widely-used \gls{ROS} framework and respects community established standards. This allows easy integration with existing planning, mapping and communication algorithms enabling quick development of multi-robot systems suited for the particular task. To my best knowledge, the presented implementation is the first implementation of a \gls{3D} map-merging algorithm for multi-robot systems within the \gls{ROS} ecosystem.

Chapter~\ref{chap:analysis} of this work discusses related works and various approaches to estimation of robot positions in an unknown environment. Chapter~\ref{chap:mergingalgorithm} introduces the presented map-merging algorithm. Chapter~\ref{chap:implementation} presents the implementation of the developed map-merging algorithm for the \gls{ROS} framework. Chapter~\ref{chap:evaluation} accesses performance of the implementation using several standard robotic datasets and experiments done by the author. Documentation for the \gls{ROS} implementation is attached as Appendix~\ref{chap:map_merge-wiki}.

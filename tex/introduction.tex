\chapter*{Introduction}
\addcontentsline{toc}{chapter}{Introduction}

Multirobot systems are established research area within robotics and artificial intelligence with growing number of applications. The multirobot system can improve effectiveness both in the term of the performance and robustness over a single robot. A team of robots can also solve tasks that can not be tackled by the single robot. Multirobot systems can scale to large environments and the most demanding applications.

Multirobot systems have been deployed to several application domains including space exploration~\cite{goldberg2002distributedspace} and \cite{huntsberger2003campout}, autonomous patrolling~\cite{parker2003parolling100}, disaster rescue and victim search, aerial surveillance, unmanned delivery, mine cleaning, snow removal~\cite{choset2001coverage}, assembly of large-scale buildings or planetary habitat~\cite{goldberg2002distributedspace} and robotic football~\cite{asada1999robocup}. Those applications require robots to operate in  dynamic environments, with high degree of uncertainty and external changes causes by other robots, humans and other agents that are not part of the multirobot system itself.

To be able to solve such demanding problems multirobot systems rely on distributed planning, communication, and control algorithms. This work presents an algorithm for estimating positions of the robots in the shared environment without any previous knowledge or assumptions about environment and for building the global map of the environment. Knowledge of positions of the robots and the map of the environment are crucial elements for effective collaborative planning and coordination. \textit{Unaware systems}~\cite{farinelli2004multirobot} that don't have knowledge of the presence of other robots are normally used only for very
simple tasks~\cite{farinelli2004multirobot}.

A typical multirobot system consists of many software parts typically structured in a multi-layer architecture. The implementation presented in this work leverages modularity of the widely-used \gls{ROS} framework and respects community established standards. This allows easy integration with existing planning, mapping and communication algorithms enabling quick development of multirobot systems suited for the particular task.

Chapter~\ref{chap:analysis} discusses related works and possible approaches to estimate robot positions in unknown environment. Chapter~\ref{chap:mergingalgorithm} introduces the implemented map-merging algorithm. Chapter~\ref{chap:implementation} presents implementation of the developed map-merging algorithm for the \gls{ROS} framework. Chapter~\ref{chap:evaluation} accesses performance of the implementation using several standard robotic datasets and experiments done by the author. Documentation for the \gls{ROS} implementation is attached as appendix~\ref{chap:map_merge-wiki}.

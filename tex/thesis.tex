%%% The main file. It contains definitions of basic parameters and includes all other parts.

%% Settings for single-side (simplex) printing
% Margins: left 40mm, right 25mm, top and bottom 25mm
% (but beware, LaTeX adds 1in implicitly)
\documentclass[12pt,a4paper]{report}
\setlength\textwidth{145mm}
\setlength\textheight{247mm}
\setlength\oddsidemargin{15mm}
\setlength\evensidemargin{15mm}
\setlength\topmargin{0mm}
\setlength\headsep{0mm}
\setlength\headheight{0mm}
% \openright makes the following text appear on a right-hand page
\let\openright=\clearpage

%% Settings for two-sided (duplex) printing
% \documentclass[12pt,a4paper,twoside,openright]{report}
% \setlength\textwidth{145mm}
% \setlength\textheight{247mm}
% \setlength\oddsidemargin{14.2mm}
% \setlength\evensidemargin{0mm}
% \setlength\topmargin{0mm}
% \setlength\headsep{0mm}
% \setlength\headheight{0mm}
% \let\openright=\cleardoublepage

%% Generate PDF/A-2u
\usepackage[a-2u]{pdfx}

%% Character encoding: usually latin2, cp1250 or utf8:
\usepackage[utf8]{inputenc}

%% Prefer Latin Modern fonts
\usepackage{lmodern}

%% Further useful packages (included in most LaTeX distributions)
\usepackage{amsmath}        % extensions for typesetting of math
\usepackage{amsfonts}       % math fonts
\usepackage{amsthm}         % theorems, definitions, etc.
\usepackage{bbding}         % various symbols (squares, asterisks, scissors, ...)
\usepackage{bm}             % boldface symbols (\bm)
\usepackage{graphicx}       % embedding of pictures
\usepackage{fancyvrb}       % improved verbatim environment
\usepackage[numbers]{natbib}         % citation style AUTHOR (YEAR), or AUTHOR [NUMBER]
\usepackage[nottoc]{tocbibind} % makes sure that bibliography and the lists
			    % of figures/tables are included in the table
			    % of contents
\usepackage{dcolumn}        % improved alignment of table columns
\usepackage{booktabs}       % improved horizontal lines in tables
\usepackage{paralist}       % improved enumerate and itemize
\usepackage[usenames]{xcolor}  % typesetting in color

%% my packages
\usepackage{algorithm}
\usepackage{algpseudocode}
\usepackage{mathtools}
\usepackage[acronym,toc]{glossaries}
\usepackage[toc,page,titletoc]{appendix}
\usepackage{subcaption}
\usepackage{dirtree}

%%% Basic information on the thesis

% Thesis title in English (exactly as in the formal assignment)
\def\ThesisTitle{Automatic Point Clouds Merging}

% Author of the thesis
\def\ThesisAuthor{Jiří Hörner}

% Year when the thesis is submitted
\def\YearSubmitted{2018}

% Name of the department or institute, where the work was officially assigned
% (according to the Organizational Structure of MFF UK in English,
% or a full name of a department outside MFF)
\def\Department{Department of Theoretical Computer Science and Mathematical Logic}

% Is it a department (katedra), or an institute (ústav)?
\def\DeptType{Department}

% Thesis supervisor: name, surname and titles
\def\Supervisor{RNDr. David Obdržálek, Ph.D.}

% Supervisor's department (again according to Organizational structure of MFF)
\def\SupervisorsDepartment{Department of Theoretical Computer Science and Mathematical Logic}

% Study programme and specialization
\def\StudyProgramme{Computer Science}
\def\StudyBranch{Artificial Intelligence}

% An optional dedication: you can thank whomever you wish (your supervisor,
% consultant, a person who lent the software, etc.)
\def\Dedication{%
I wish to thank to Univ.Prof. Dr. Stephan Weiss and Dejan Sazdov from Alpen-Adria-Universität Klagenfurt for kindly providing data for one of the datasets. I wish to thank to my family for their support.
}

% Abstract (recommended length around 80-200 words; this is not a copy of your thesis assignment!)
\def\Abstract{%
Multi-robot systems are an established research area with a growing number of applications. Efficient coordination in such systems usually requires knowledge of robot positions and the global map. This work presents a novel map-merging algorithm for merging 3D point cloud maps in multi-robot systems, which produces the global map and estimates robot positions. The algorithm is based on feature-matching transformation estimation with a novel descriptor matching scheme and works solely on point cloud maps without any additional auxiliary information. The algorithm can work with different SLAM approaches and sensor types and it is applicable in heterogeneous multi-robot systems. The map-merging algorithm has been evaluated on real-world datasets captured by both aerial and ground-based robots with a variety of stereo rig cameras and active RGB-D cameras. It has been evaluated in both indoor and outdoor environments. The proposed algorithm was implemented as a ROS package and it is currently distributed in the ROS distribution. To the best of my knowledge, it is the first ROS package for map-merging of 3D maps.
}

% 3 to 5 keywords (recommended), each enclosed in curly braces
\def\Keywords{%
{ROS} {map-merging} {Point cloud} {multi-robot systems}
}

%% The hyperref package for clickable links in PDF and also for storing
%% metadata to PDF (including the table of contents).
%% Most settings are pre-set by the pdfx package.
\hypersetup{unicode}
\hypersetup{breaklinks=true}

% create a glossary
\makeglossaries
% Definitions of acronyms
%%% Abbreviations used in the thesis, including their explanation

\newacronym{RANSAC}{RANSAC}{Random Sample Consensus}
\newacronym{BFS}{BFS}{Breadth-First Search}
\newacronym{DFS}{DFS}{Depth-First Search}
\newacronym{ROS}{ROS}{Robot Operating System}
\newacronym{OpenCV}{OpenCV}{Open Source Computer Vision Library}
\newacronym{SIFT}{SIFT}{Scale-Invariant Feature Transform}
\newacronym{SLAM}{SLAM}{Simultaneous Localization and Mapping}
\newacronym{FLANN}{FLANN}{Fast Library for Approximate Nearest Neighbors}
\newacronym{MIT}{MIT}{Massachusetts Institute of Technology}
\newacronym{GPS}{GPS}{Global Positioning System}
\newacronym{ICP}{ICP}{Iterative Closest Point}
\newacronym{AAU}{AAU}{Alpen-Adria-Universität Klagenfurt}
\newacronym{PFH}{PFH}{Point Feature Histogram}
\newacronym{PFHRGB}{PFHRGB}{Point Feature Histogram with colour}
\newacronym{FPFH}{FPFH}{Fast Point Feature Histogram}
\newacronym{SHOT}{SHOT}{Signature of Histograms of Orientations}
\newacronym{RSD}{RSD}{Radius-based Surface Descriptor}
\newacronym{SC3D}{SC3D}{3D Shape Context}
\newacronym{SAC-IA}{SAC-IA}{Sample Consensus Initial Alignment}
\newacronym{SVD}{SVD}{Singular-Value Decomposition}
\newacronym{API}{API}{Application Programming Interface}
\newacronym{PCL}{PCL}{Point Cloud Library}
\newacronym{NARF}{NARF}{Normal Aligned Radial Feature}
\newacronym{RGB}{RGB}{Red-Green-Blue}
\newacronym{RGB-D}{RGB-D}{Red-Green-Blue-Depth}
\newacronym{3D}{3D}{three-dimensional}
\newacronym{2D}{2D}{two-dimensional}
\newacronym{IMU}{IMU}{Inertial Measurement Unit}
\newacronym{EKF}{EKF}{Extended Kalman Filter}


% Definitions of macros (see description inside)
%%% This file contains definitions of various useful macros and environments %%%
%%% Please add more macros here instead of cluttering other files with them. %%%

%%% Minor tweaks of style

% These macros employ a little dirty trick to convince LaTeX to typeset
% chapter headings sanely, without lots of empty space above them.
% Feel free to ignore.
\makeatletter
\def\@makechapterhead#1{
  {\parindent \z@ \raggedright \normalfont
   \Huge\bfseries \thechapter. #1
   \par\nobreak
   \vskip 20\p@
}}
\def\@makeschapterhead#1{
  {\parindent \z@ \raggedright \normalfont
   \Huge\bfseries #1
   \par\nobreak
   \vskip 20\p@
}}
\makeatother

% This macro defines a chapter, which is not numbered, but is included
% in the table of contents.
\def\chapwithtoc#1{
\chapter*{#1}
\addcontentsline{toc}{chapter}{#1}
}

% Draw black "slugs" whenever a line overflows, so that we can spot it easily.
\overfullrule=1mm

%%% Macros for definitions, theorems, claims, examples, ... (requires amsthm package)

\theoremstyle{plain}
\newtheorem{thm}{Theorem}
\newtheorem{lemma}[thm]{Lemma}
\newtheorem{claim}[thm]{Claim}

\theoremstyle{plain}
\newtheorem{defn}{Definition}

\theoremstyle{remark}
\newtheorem*{cor}{Corollary}
\newtheorem*{rem}{Remark}
\newtheorem*{example}{Example}

%%% An environment for proofs

%%% FIXME %%% \newenvironment{proof}{
%%% FIXME %%%   \par\medskip\noindent
%%% FIXME %%%   \textit{Proof}.
%%% FIXME %%% }{
%%% FIXME %%% \newline
%%% FIXME %%% \rightline{$\square$}  % or \SquareCastShadowBottomRight from bbding package
%%% FIXME %%% }

%%% An environment for typesetting of program code and input/output
%%% of programs. (Requires the fancyvrb package -- fancy verbatim.)

\DefineVerbatimEnvironment{code}{Verbatim}{fontsize=\small, frame=single}

%%% The field of all real and natural numbers
\newcommand{\R}{\mathbb{R}}
\newcommand{\N}{\mathbb{N}}

%%% Useful operators for statistics and probability
\DeclareMathOperator{\pr}{\textsf{P}}
\DeclareMathOperator{\E}{\textsf{E}\,}
\DeclareMathOperator{\var}{\textrm{var}}
\DeclareMathOperator{\sd}{\textrm{sd}}

%%% Transposition of a vector/matrix
\newcommand{\T}[1]{#1^\top}

%%% Various math goodies
\newcommand{\goto}{\rightarrow}
\newcommand{\gotop}{\stackrel{P}{\longrightarrow}}
\newcommand{\maon}[1]{o(n^{#1})}
\newcommand{\abs}[1]{\left|{#1}\right|}
\newcommand{\dint}{\int_0^\tau\!\!\int_0^\tau}
\newcommand{\isqr}[1]{\frac{1}{\sqrt{#1}}}

%%% Various table goodies
\newcommand{\pulrad}[1]{\raisebox{1.5ex}[0pt]{#1}}
\newcommand{\mc}[1]{\multicolumn{1}{c}{#1}}

%%% My macros

\DeclareMathOperator{\rank}{rank}
\DeclareMathOperator{\sgn}{sgn}
\DeclareMathOperator{\trace}{trace}
\DeclareMathOperator{\conv}{conv}
%\DeclareMathOperator{\R}{\mathbb{R}}
\DeclareMathOperator{\bigO}{\mathcal{O}}
\DeclarePairedDelimiter{\ev}{\operatorname{E}[}{]}
\DeclarePairedDelimiter{\prob}{\operatorname{P}[}{]}
%\DeclarePairedDelimiter{\abs}{\lvert}{\rvert}
\DeclarePairedDelimiter{\ceil}{\lceil}{\rceil}
\DeclarePairedDelimiter{\floor}{\lfloor}{\rfloor}
\DeclarePairedDelimiter{\norm}{\lVert}{\rVert}
\DeclarePairedDelimiter{\scal}{\langle}{\rangle}

\renewcommand{\algorithmicrequire}{\textbf{Input:}}
\renewcommand{\algorithmicensure}{\textbf{Output:}}

% ROS node API formatting
\newcommand{\ROStopic}[3]{\begin{sloppypar} \noindent\texttt{#1} (#2) \end{sloppypar} \par \hangindent=15pt \hangafter=0 \noindent #3}
\newcommand{\ROSparam}[4]{\begin{sloppypar} \noindent\texttt{#1} (#3, default: \texttt{#2}) \end{sloppypar} \par \hangindent=15pt \hangafter=0 \noindent #4}
\newcommand{\ROStransform}[3]{\begin{sloppypar} \noindent\texttt{#1} $\rightarrow$ \texttt{#2} \end{sloppypar} \par \hangindent=15pt \hangafter=0 \noindent #3}

% allow hyphenation after underscore
\let\oldunderscore\_
\renewcommand{\_}{\oldunderscore\-}

\hyphenation{mer-ged}
\hyphenation{cost-map}
\hyphenation{data-set}
% \hyphenation{sen-sor_-msgs/-Point-Cloud2}
\hyphenation{PFH-RGB}

\graphicspath {{../img/plots/}}


% Title page and various mandatory informational pages
\begin{document}
\include{title}

%%% A page with automatically generated table of contents of the master thesis

\tableofcontents

%%% Each chapter is kept in a separate file
\chapter*{Introduction}
\addcontentsline{toc}{chapter}{Introduction}

Multirobot systems are established research area within robotics and artificial intelligence with growing number of applications. A multirobot system can improve effectiveness in terms of both performance and robustness over a single robot. A team of robots can also solve tasks that can not be tackled by a single robot. Multirobot systems can scale to large environments and to the most demanding applications.

Multirobot systems have been deployed to several application domains including space exploration~\citep{goldberg2002distributedspace,huntsberger2003campout}, autonomous patrolling~\citep{parker2003parolling100}, disaster rescue and victim search, aerial surveillance, unmanned delivery, mine cleaning, snow removal~\citep{choset2001coverage}, assembly of large-scale buildings or planetary habitat~\citep{goldberg2002distributedspace} and robotic football~\citep{asada1999robocup}. Those applications require robots to operate in  dynamic environments, with high degree of uncertainty and external changes caused by other robots, humans and other agents that are not part of the multirobot system itself.

To be able to solve such demanding problems multirobot systems rely on distributed planning, communication and control algorithms. This work presents an algorithm for estimating positions of the robots in the shared environment and for building the global map of the environment (map-merging) without any previous knowledge or assumptions about the environment. Knowledge of positions of the robots and the map of the environment are crucial elements for effective collaborative planning and coordination. While multirobot systems that don't have knowledge of the presence of other robots has been developed, \citet{farinelli2004multirobot} lists such systems as \textit{unaware systems}, these systems are normally used only for very simple tasks~\citep{farinelli2004multirobot}. For most of the multirobot systems knowledge of positions of robots is essential and map-merging is a key problem to solve.

A typical multirobot system consists of many software parts typically structured in a multi-layer architecture. The implementation presented in this work leverages modularity of the widely-used \gls{ROS} framework and respects community established standards. This allows easy integration with existing planning, mapping and communication algorithms enabling quick development of multirobot systems suited for the particular task.

Chapter~\ref{chap:analysis} of this work discusses related works and various approaches to estimation of robots positions in unknown environment. Chapter~\ref{chap:mergingalgorithm} introduces the presented map-merging algorithm. Chapter~\ref{chap:implementation} presents implementation of the developed map-merging algorithm for the \gls{ROS} framework. Chapter~\ref{chap:evaluation} accesses performance of the implementation using several standard robotic datasets and experiments done by the author. Documentation for the \gls{ROS} implementation is attached as appendix~\ref{chap:map_merge-wiki}.

\chapter{Analysis}
\label{chap:analysis}

\cite{Andel07}

\chapter{Merging algorithm}
\label{chap:mergingalgorithm}

% TODO add visualisation

\begin{algorithm}
    \caption[Pair-wise transformation estimation]{Estimates pair-wise transformation between two points-clouds}
    \label{alg:estimate-pair}
    \begin{algorithmic}[1]
        \Require $2$ maps represented as pointclouds
        \Ensure transformation estimate between $2$ maps
        \Procedure{estimateTransform}{$map1, map2$}
            \State down-sample to working resolution
            \State remove outliers
            \State detect keypoints
            \State detect normals
            \State compute descriptor for each keypoint
            \State match descriptors and compute initial transformation
            \State refine transformation with \gls{ICP}
        \EndProcedure
    \end{algorithmic}
\end{algorithm}

This section presents an algorithm for estimating transformations between $n$ pointclouds and merging them together. As discussed in the chapter~\ref{chap:analysis}, we work with maps represented as pointclouds, possibly with RGB information for each point.

There are two core problems for estimating the transformation. First we need to able to estimate pair-wise transformation for $2$ maps using only geometrical and possibly colour information available within pointclouds. We discuss our method in section~\ref{sec:estimate-pair-wise}. Second, we want to get a transformation for each of the maps to the selected reference frame. This is discussed in section~\ref{sec:estimate-global}.

After we have estimated the transformations we can stitch them to create the global map.

\section{Estimating pair-wise transformation}
\label{sec:estimate-pair-wise}

Algorithm~\ref{alg:estimate-pair} describes a pipeline of pointcloud algorithms to estimate the pair-wise transformation. Every step is discussed in the following sections.

\subsection{Down-sampling}

As we are working with possibly large-scale maps, input pointclouds may contain millions of points. To reduce computation times it is highly desirable to reduce number of points. As discussed in section~\ref{chap:analysis}, in multi-robot system this step is typically performed by \gls{SLAM} running on each robot before publishing the map, thus saving bandwidth of the communication. However for the purpose of transformation estimation, we might want reduce resolution even further to reduce computation time (for example each robot might publish a map with typical resolution $0.05 m/\text{voxel}$, but for estimation we might work with resolution $0.1 m/\text{voxel}$).

We show is the section~\ref{chap:evaluation} that our algorithm can reliable estimate the transformation in just $0.1 m/\text{voxel}$ resolution.

Common technique for reducing resolution of pointcloud is voxelisation, which produces a voxel grid. The voxel grid is a regularly spaced, three-dimensional grid. We can represent the voxel grid as a normal pointcloud, with each point representing a voxel of the voxel grid. We don't usually save empty space information, so the grid is sparse.

Algorithm for voxelisation is simple. For each voxel (size of the voxel is determined by the resolution) we take all points contained in the voxel and approximate them with their centroid.

\subsection{Removing outliers}

Although the voxelisation can deal some some of the noise and inaccuracies, during experiments it has been beneficial to perform further outliers filtering to remove far laying outliers. Far-laying outliers may end-up being detected as keypoints, but they are usually not matched. Reducing the number of detected keypoints speeds up the later phases of estimation.

I have selected to use a simple radius-based outlier removal. This method searches for neighbour of each point with certain radius and removes points that have neighbours count below certain threshold.

Because the radius outlier removal removes only the points with few close neighbours, it does not reduce the robustness of the estimation. Because descriptors of the outlier points are based on just a few points, they didn't produced a robust matching candidates during experiments.

For example on $2$ maps from \gls{AAU} dataset (see section~\ref{sec:aau-dataset}), outlier removal removes $326$ and $271$ points respectively ($7.29\%$, $6.16\%$), but number of detected keypoints decreases from $66, 63$ to $51, 56$ ($22.73\%$, $11.11\%$). Outlier removal didn't impact the estimation process negatively.

\subsection{Estimating surface normals}

The last preprocessing step is to estimate surface normals. Surface normals are used in later steps to compute descriptors and by Harris keypoint detector.

Algorithm for estimating surface normals is described in~\cite{RusuDoctoralDissertation}. The most important parameter for normals estimation is the size of the neighbourhood which is used for the estimation. This can be configured by the user.

\subsection{Detecting keypoints}

% TODO we need to detect keypoints

There are two families of keypoints detectors that are being used with pointclouds. Most of the approaches has been adapted from keypoints detectors that has been originally developed to work on images.

First class of detectors uses RGB colour information stored for each point in the pointcloud. This approach suppose that the pointcloud has been obtained with detector that provide the colour information such as stereo rig camera setup, active RGB-D cameras etc. For our approach we use \gls{SIFT} keypoint detector, which is an adapted algorithm from~\cite{lowe2004distinctive} that works on pointclouds with RGB information.

Second class of algorithms works with just geometrical information and is therefore able to work with pointclouds that do not store any additional information for points. These algorithms can be used with pointclouds composed of laser scans. Our algorithm implements Harris 3D keypoint detector for this purpose, which is an adapted algorithm from~\cite{harris1988combined}. Instead of using image gradients, which are not available in the pointcloud without colour information, it uses surface normals, that capture geometrical properties of the point neighbourhood.

\subsection{Computing descriptors}



\section{Estimating global transformations}
\label{sec:estimate-global}
\chapter{Implementation}
\label{chap:implementation}

I have implemented the map-merging algorithm (chapter~\ref{chap:mergingalgorithm}) in the \gls{ROS} framework (section~\ref{sec:ros}). This allows the implementation to be used with various readily available \gls{SLAM} algorithms and inter-robot communications.

Pointcloud algorithms are implemented with \gls{PCL} (section~\ref{sec:pcl}). \gls{PCL} is a popular open-source library for manipulating point-cloud data.

\section{Robot Operating System}
\label{sec:ros}

\gls{ROS} is a popular open-source robotics middleware and distribution of community-maintained robotics software. \gls{ROS} provides hardware abstraction, low-level sensors control, implementations of commonly-used robotics algorithms, including several variants of \gls{SLAM}, message-passing between processes, and package and build management.

Core of the \gls{ROS} is a messaging system, that allows loose coupling between sensors, robotic algorithms and actuators. There are many community-established standard message formats for various sensor types, visualisation, maps, robot state, linear algebra etc., which allows software reuse.

Software in the \gls{ROS} project is organised in the form of community-maintained packages. First type of packages integrate particular hardware sensors, actuators or entire robot platforms with \gls{ROS} and provide a message interface using standard \gls{ROS} messages. Second type of packages implement robotics algorithms, that usually depend only on standard \gls{ROS} messages, making them reusable between robots. The \gls{ROS} project builds and distributes binary packages for Linux.

\section{\texttt{map\_merge\_3d} package}

\subsection{Communication}

\subsection{Map representation}

\subsection{Configuration}

\subsection{Estimation}

% \section{Point Cloud Library}
% \label{sec:pcl}

% \gls{PCL} is an open-source library for pointcloud processing. It contains routines for
\chapter{Evaluation}
\label{chap:evaluation}

The presented map-merging algorithm (chapter~\ref{chap:mergingalgorithm}) has been evaluated on number of demanding robotics datasets. The datasets include data captured by small aerial vehicles (sections~\ref{sec:euroc-dataset}, \ref{sec:aau-dataset}) as well as ground-based robots. The datasets include both widely used benchmark datasets in robotics research as well as data recorded by the author.

Sensors used include stereo rig cameras, active \gls{RGB-D} cameras and laser scans? This variety of sensors and robots covers many typical multi-robot deployments. All datasets has been captured under real-word conditions, none of them uses simulated data.

The evaluation focuses on properties of the presented pair-wise transformation estimation algorithm for pointclouds (section~\ref{sec:estimate-pair-wise}, which is the core algorithm of the map-merging \gls{ROS} node (section~\ref{sec:ros-package}. Accuracy of the estimation algorithm is critical for the overall map-merging process.

\section{The EuRoC micro aerial vehicle datasets}
\label{sec:euroc-dataset}

The dataset introduced by~\citet{Burri2016} was collected on-board a micro aerial vehicle equipped with stereo camera rig and \gls{IMU}. Calibration data for the cameras and ground-truth data are provided with the dataset. This dataset has been used extensively for evaluation of the visual \gls{SLAM} algorithms and visual odometry approaches.

The cameras produce a WVGA monochrome (greyscale) images at $20$ frames per second. Cameras have a global shutter. The automatic exposure control is independent for both cameras. According to the published errata~\citep{Burri2016}, this resulted in different shutter times and in turn in different image brightnesses, rendering stereo matching and feature tracking more challenging.

The dataset contains $11$ mapping sessions in $3$ different environments. Each mapping session is available in a single \gls{ROS} bag file. First $5$ sessions were captured in ETH machine hall (figure~\ref{fig:eth-machine-hall}), a fairly large industrial environment featuring piping, reservoirs and many different types of surfaces. Second and third batch of datasets were captured in a smaller furnished rectangular room. For the second and the third batch the furnishing was different.

\begin{figure}
    \centering
    \includegraphics[width=\textwidth]{../img/eth_machine_room.jpg}
    \caption[ETH Machine hall]{ETH Machine hall industrial environment where $5$ mapping sessions of EuRoC dataset were captured. The image is courtesy of the authors~\citep{Burri2016}.}
    \label{fig:eth-machine-hall}
\end{figure}

The dataset is intended for evaluating \gls{SLAM} algorithms, for our purposes it was necessary to process the data with a \gls{SLAM} algorithm to create a pointcloud maps. First, I have used the provided calibration sequence and I have created a calibration data for \gls{ROS} using the \texttt{camera\_calibration} tool available with \gls{ROS}.

Second, for each environment I created a pair of maps using the $01$ and $02$ mapping sessions from the datasets, which were used further for the evaluation of the map-merging. I used a RTAB-Map \gls{SLAM}, developed by~\citet{labbe2014online}, to create the maps. The odometry for mapping was provided from stereo camera data, using a visual odometry approach, the available \gls{IMU} data were not used.

% finish diffucult maps, articats, differnt visual odometry

% todo add maps figures

\section{AAU dataset}
\label{sec:aau-dataset}

\section{MFF dataset}
\label{sec:mff-dataset}
\chapter*{Conclusion}
\addcontentsline{toc}{chapter}{Conclusion}

This work presented a novel map-merging algorithm for merging \gls{3D} point cloud maps in multi-robot systems. The algorithm is based on feature-matching transformation estimation and works solely on point cloud maps without any additional auxiliary information. This makes the algorithm applicable in heterogeneous multi-robot systems and the algorithm can work with different \gls{SLAM} approaches and sensor types. To the best of my knowledge, the presented approach is the first implemented map-merging algorithm working directly on point clouds without any extra information.

The work showed the feasibility of the feature-matching approach for registration of low-density point cloud maps produced by \gls{SLAM} algorithms while using \gls{3D} point cloud features typically employed with high-density sensor data.

A reciprocal descriptor matching algorithm was introduced for estimating the initial transformation using feature-matching. The algorithm requires very little parametrisation and exhibited good performance across descriptors in the evaluation. In the most configurations, it outperforms \gls{SAC-IA} algorithm for initial alignment available in the \gls{PCL} library.

The map-merging algorithm has been evaluated on real-world datasets captured by both aerial and ground-based robots with a variety of stereo rig cameras and active \gls{RGB-D} cameras. It has been evaluated in both indoor and outdoor environments ranging from forest to a single furnished room. The datasets used for evaluation include both well-established benchmark robotics datasets as well as my own experiments.

The proposed algorithm was implemented as a \gls{ROS} package. To the best of my knowledge, it is the first \gls{ROS} package for map-merging of \gls{3D} maps. The package has been submitted to the \gls{ROS} distribution, the binary packages are distributed with the \gls{ROS} since the \gls{ROS} Melodic Morenia release. The implementation does not require any particular communication solution between robots and can work with \gls{ROS} in both multi-master and single-master setup. Likewise, the implementation does not presume any particular \gls{SLAM} method, nor any particular sensor and uses a portable point cloud map representation, which makes it compatible with existing readily available \gls{SLAM} implementations.

While the selected map representation enables great interoperability with existing software, the monolithic point clouds do not permit efficient repairing of mapping errors in the merged map. A pose graph of point clouds representation would be beneficial for the map-merging, but there is no standardised message format in the \gls{ROS} for such a representation nor there is a common graph representation established across different \gls{SLAM} implementations. In the future it would be beneficial to introduce a portable pose graph representation to the \gls{ROS}, as discussed in Section~\ref{sec:map-representation}, support it within the core \gls{ROS} packages and promote its usage across \gls{SLAM} implementations. This representation would allow the presented algorithm to work on sub-maps in the pose graph and repair the mapping errors in the merged map.


%%% Bibliography
\include{bibliography}

%%% Figures used in the thesis (consider if this is needed)
\listoffigures

%%% Tables used in the thesis (consider if this is needed)
%%% In mathematical theses, it could be better to move the list of tables to the beginning of the thesis.
% \listoftables

\listofalgorithms
\addcontentsline{toc}{chapter}{List of Algorithms}

%%% Abbreviations used in the thesis, if any, including their explanation
%%% In mathematical theses, it could be better to move the list of abbreviations to the beginning of the thesis.
% \chapwithtoc{List of Abbreviations}
\printglossary[title=List of Abbreviations,type=\acronymtype]

%%% Attachments to the master thesis, if any. Each attachment must be
%%% referred to at least once from the text of the thesis. Attachments
%%% are numbered.
%%%
%%% The printed version should preferably contain attachments, which can be
%%% read (additional tables and charts, supplementary text, examples of
%%% program output, etc.). The electronic version is more suited for attachments
%%% which will likely be used in an electronic form rather than read (program
%%% source code, data files, interactive charts, etc.). Electronic attachments
%%% should be uploaded to SIS and optionally also included in the thesis on a~CD/DVD.
%%% Allowed file formats are specified in provision of the rector no. 72/2017.
% \appendix
% \chapter{Attachments}

% list of attached files
\chapwithtoc{List of Attached Files}
\label{chap:files}

This is a list of files in the electronic attachment to this work. The attachment contains source code for the presented \gls{ROS} package (Chapter~\ref{chap:implementation}), the package documentation and experimental data (Chapter~\ref{chap:evaluation}). Source code for the \gls{ROS} package is also available online (\url{https://github.com/hrnr/map-merge}). A reproduction of the package documentation is attached in the printed version (Appendix~\ref{chap:map_merge-wiki}).

\medskip
\dirtree{%
.1 attachments.zip. %
.2 evaluation\DTcomment{datasets used in evaluation}. %
.3 README.md. %
.3 AAU\DTcomment{Section~\ref{sec:aau-dataset}}. %
.4 AAU forest 2.pcd. %
.4 AAU forest 1.pcd. %
.3 EuRoC\DTcomment{Section~\ref{sec:euroc-dataset}}. %
.4 Vicon Room 1 02.pcd. %
.4 Vicon Room 1 01.pcd. %
.4 Machine Hall 02.pcd. %
.4 Machine Hall 01.pcd. %
.4 Vicon Room 2 01.pcd. %
.4 Vicon Room 2 02.pcd. %
.3 MFF\DTcomment{Section~\ref{sec:mff-dataset}}. %
.4 MFF Rotunda 1.pcd. %
.4 MFF Refectory 1.pcd. %
.4 MFF Refectory 2.pcd. %
.4 MFF Rotunda 2.pcd. %
.2 map-merge\DTcomment{sources for the \gls{ROS} package}. %
.3 LICENSE. %
.3 README.md. %
.3 map\_merge\_3d\DTcomment{Section~\ref{sec:ros-package}}. %
.4 rosdoc.yaml. %
.4 CMakeLists.txt. %
.4 package.xml. %
.4 CHANGELOG.rst. %
.4 src. %
.5 visualise.cpp. %
.5 graph.cpp. %
.5 visualise.h. %
.5 map\_merging.cpp. %
.5 dispatch\_descriptors.h. %
.5 registration\_visualisation.cpp. %
.5 map\_merge\_node.cpp. %
.5 matching.cpp. %
.5 map\_merge\_tool.cpp. %
.5 graph.h. %
.5 features.cpp. %
.4 launch. %
.5 map\_merge\_3d.rviz. %
.5 from\_pcds.launch. %
.5 map\_merge.launch. %
.4 include. %
.5 map\_merge\_3d. %
.6 typedefs.h. %
.6 map\_merging.h. %
.6 matching.h. %
.6 map\_merge\_node.h. %
.6 features.h. %
.6 enum.h. %
.4 test. %
.5 test\_map\_merging.cpp. %
.4 doc. %
.5 mainpage.dox. %
.5 architecture.svg. %
.5 screenshot.jpg. %
.5 wiki.txt. %
.5 architecture.dia. %
.2 doc\DTcomment{generated code documentation for the \gls{ROS} package}. %
.3 html. %
.3 manifest.yaml. %
}


\begin{appendices}
\chapter{map\_merge\_3d}
\label{chap:map_merge-wiki}

This is a reproduction of the text available online at \url{http://wiki.ros.org/map_merge_3d}. Although maintained as the wiki the current version of the text reproduced below has been written solely by the author.

\section{Package Summary}

Merging multiple \gls{3D} maps, represented as pointclouds, without knowledge of initial positions of robots.

\begin{itemize}
    \item Maintainer status: developed
    \item Maintainer: Jiri Horner \textless laeqten AT gmail DOT com\textgreater
    \item Author: Jiri Horner \textless laeqten AT gmail DOT com\textgreater
    \item License: BSD
    \item Source: git \url{https://github.com/hrnr/map-merge.git} (branch: lunar-devel)
\end{itemize}


\section{Overview}

This package provides a \gls{3D} global map for multiple robots and the respective transformations between robots. It merges robots' individual maps based on the overlapping space in the maps and requires no dependencies on a particular \gls{SLAM} or communication between the robots.

\begin{figure}
    \centering
    \includegraphics[width=4.53in]{../img/screenshot.jpg}
    \caption[The merged map for $2$ robots.]{Visualisation of registration between $2$ maps using a \texttt{map\_merge\_3d} package.}
    \label{fig:mapmergescreenshot}
\end{figure}

The \gls{ROS} node can merge maps from the arbitrary number of robots. It expects maps from individual robots as \gls{ROS} topics and does not impose any particular messaging between robots. If your run multiple robots under the same \gls{ROS} master then \texttt{map\_merge\_3d} may work for you out-of-the-box, this makes it easy to setup a simulation experiment.

In the multi-robot exploration scenario your robots probably run multiple \gls{ROS} masters and you need to setup a communication link between robots. Common solution might be \href{http://wiki.ros.org/multimaster_fkie}{multimaster\_fkie} package. You need to provide maps from your robots on local topics (under the same master). Also if you want to distribute merged map and \href{http://wiki.ros.org/tf}{tf} transformations back to robots your communication must take care of it.

\section{Architecture}

\texttt{map\_merge\_3d} finds robot maps automatically and new robots can be added to the system at any time. \gls{3D} maps are expected as \texttt{sensor\_msgs/PointCloud2}, other map messages are not supported.

\begin{figure}
    \centering
    \includegraphics[width=\textwidth]{../img/architecture.pdf}
    \caption[The architecture of the \texttt{map\_merge\_node}]{Diagram showing \gls{ROS} \gls{API} of the map-merging node.}
    \label{fig:architecture}
\end{figure}

Recommended topics names for robot maps are \texttt{/robot1/map}, \texttt{/robot2/map} etc. However the names are configurable. All robots are expected to publish map under \texttt{<robot\_namespace>/map}, where topic name (\texttt{map}) is configurable, but must be the same for all robots. For each robot \texttt{<robot\_namespace>} is of cause different, but it does not need to follow any pattern. Further, you can exclude some topics using \texttt{robot\_namespace} parameter, to avoid merging unrelated pointclouds.

\section{Estimation}

Transformations between maps are estimated by feature-matching algorithm and therefore it is required to have sufficient amount of overlapping space between maps to make a high-probability match. If maps don't have enough overlapping space to make a solid match, merger might reject those matches.

Estimating transforms between maps is cpu-intesive so you might want to tune \texttt{estimation\_rate} parameter to run the re-estimation less often.

\section{ROS API}

\subsection{map\_merge\_node}

Provides map merging services offered by this package. Dynamically looks for new robots in the system and merges their maps. Provides \texttt{tf} transforms.

\subsubsection{Subscribed Topics}

\ROStopic{<robot\_namespace>/map}{sensor\_msgs/PointCloud2}{Local map for a specific robot.}

\subsubsection{Published Topics}

\ROStopic{map}{sensor\_msgs/PointCloud2}{Merged map from all robots in the system.}

\subsubsection{Parameters}
\paragraph{Node Parameters}

Parameters affecting general setup of the node.

\ROSparam{\~{}robot\_map\_topic}{map}{string}{Name of robot map topic without namespaces (last component of the topic name). Only topics with this name are considered when looking for new maps to merge. This topics may be subject to further filtering (see below).}

\ROSparam{\~{}robot\_namespace}
{<empty string>}
{string}
{Fixed part of the robot map topic. You can employ this parameter to further limit which topics are considered during dynamic lookup for robots. Only topics which contain (anywhere) this string are considered for lookup. Unlike \texttt{robot\_map\_topic} you are not limited by namespace logic. Topics are filtered using text-based search. Therefore \texttt{robot\_namespace} does not need to be a \gls{ROS} namespace, but it can contain slashes etc. This string must be a common part of all maps topic name (all robots for which you want to merge map).}

\ROSparam{\~{}merged\_map\_topic}
{map}
{string}
{Topic name where merged map is published.}

\ROSparam{\~{}world\_frame}
{world}
{string}
{Frame id (in \href{http://wiki.ros.org/tf}{tf} tree) which is assigned to published merged map and used as reference frame for tf transforms.}

\ROSparam{\~{}compositing\_rate}
{0.3}
{double}
{Rate in Hz. Basic frequency on which the node merges maps and publishes merged map. Increase this value if you want faster updates.}

\ROSparam{\~{}discovery\_rate}
{0.05}
{double}
{Rate in Hz. Frequency on which this node discovers new robots (maps). Increase this value if you need more agile behaviour when adding new robots.}

\ROSparam{\~{}estimation\_rate}
{0.01}
{double}
{Rate in Hz. Frequency on which this node re-estimates transformations between maps. Estimation is cpu-intensive, so you may wish to lower this value.}

\ROSparam{\~{}publish\_tf}
{true}
{bool}
{Whether to publish estimated transforms in the \href{http://wiki.ros.org/tf}{tf} tree. See below.}

\paragraph{Registration Parameters}
\label{sec:registration-param}

Parameters affecting only registration algorithm used for estimating transformations between maps. These parameters should be defined in the same namespace as normal node parameters.

\ROSparam{\~{}resolution}
{0.1}
{double}
{Resolution used for the registration. Small value increases registration time.}

\ROSparam{\~{}descriptor\_radius}
{resolution * 8.0}
{double}
{Radius for descriptors computation.}

\ROSparam{\~{}outliers\_min\_neighbours}
{50}
{int}
{Minimum number of neighbours for a point to be kept in the map during outliers pruning.}

\ROSparam{\~{}normal\_radius}
{resolution * 6.0}
{double}
{Radius used for estimating normals.}

\ROSparam{\~{}keypoint\_type}
{SIFT}
{string}
{Type of keypoints used. Possible values are \gls{SIFT}, HARRIS.}

\ROSparam{\~{}keypoint\_threshold}
{5.0}
{double}
{Keypoints with lower response that this value are pruned. Lower this threshold when using Harris keypoints (you can set $0.0$).}

\ROSparam{\~{}descriptor\_type}
{PFH}
{string}
{Type of descriptors used. Possible values are \gls{PFH}, \gls{PFHRGB}, \gls{FPFH}, \gls{RSD}, \gls{SHOT}, \gls{SC3D}.}

\ROSparam{\~{}estimation\_method}
{MATCHING}
{string}
{Type of descriptors matching algorithm used. This algorithm is used for initial global match. Possible values are \texttt{MATCHING}, \texttt{SAC\_IA}.}

\ROSparam{\~{}refine\_transform}
{true}
{bool}
{Whether to refine estimated transformation with \gls{ICP} or not.}

\ROSparam{\~{}inlier\_threshold}
{resolution * 5.0}
{double}
{Inlier threshold used in \gls{RANSAC} during estimation.}

\ROSparam{\~{}max\_correspondence\_distance}
{inlier\_threshold * 2.0}
{double}
{Maximum distance for matched points to be considered the same point.}

\ROSparam{\~{}max\_iterations}
{500}
{int}
{Maximum iterations for \gls{RANSAC}.}

\ROSparam{\~{}matching\_k}
{5}
{int}
{Number of the nearest descriptors to consider for matching.}

\ROSparam{\~{}transform\_epsilon}
{1e-2}
{double}
{The smallest change allowed until \gls{ICP} convergence.}

\ROSparam{\~{}confidence\_threshold}
{0.0}
{double}
{Minimum confidence in the pair-wise transform estimate to be included in the map-merging graph. Pair-wise transformations with lower confidence are not considered when computing global transforms. Increase this value if you are having problems with invalid transforms being estimated. The confidence value is computed as a reciprocal of Euclidean distance between transformed maps.}

\ROSparam{\~{}output\_resolution}
{0.05}
{double}
{Resolution of the merged global map.}

\subsubsection{Provided tf Transforms}

\ROStransform{world}{mapX\_frame}
{Transformation from the world frame (which name can be configured using \texttt{world\_frame} parameter) to each of the maps. Each map must have a correct \texttt{frame\_id} set (instead \texttt{mapX\_frame}) in the \texttt{sensor\_msgs/PointCloud2} message. If the transformation could not be estimated, null transformation is published.}

\section{Tools}

Alongside \gls{ROS} node \texttt{map\_merge\_3d} provides command-line tools to work with pointcloud maps saved in \texttt{pcd} files. Both tools accept any of the registration parameters described in section~\ref{sec:registration-param}.

The tools use \gls{PCL} command-line parsing module. \gls{PCL} command-line parsing has some limits (\gls{PCL} users won't be surprised): it supports only \texttt{--param value} format, \texttt{--param=value} is not accepted. Unknown options are ignored. Options may be arbitrarily mixed with filenames. There are no short versions for parameters.

\subsection{map\_merge\_tool}

Tool for merging maps offline. Produces \texttt{output.pcd} with merged global map. This tool can merge arbitrary number of maps.

\subsubsection{Usage}

\begin{code}
rosrun map_merge_3d map_merge_tool [--param value] map1.pcd
map2.pcd [map3.pcd...]
\end{code}

For example to use SHOT descriptors with 3 maps:

\begin{code}
rosrun map_merge_3d map_merge_tool --descriptor_type SHOT map1.pcd
map2.pcd map3.pcd
\end{code}

\subsection{registration\_visualisation}

Visualises pair-wise transform estimation between 2 maps. Uses \gls{PCL} visualiser for the visualisation.

\subsubsection{Usage}

\begin{code}
rosrun map_merge_3d registration_visualisation [--param value]
map1.pcd map2.pcd
\end{code}

After one step of the estimation a visualisation window appears. You can freely navigate the pointcloud, save a screenshot or camera parameters (press \texttt{h} to see all shortcuts). After the window is closed, estimation continues with the next phase and the next visualisation window appears. Details about estimation progress are printed to \texttt{stdout}.

\section{Acknowledgements}

This package was developed as part of my master thesis at \href{http://www.mff.cuni.cz/to.en/}{Charles University} in Prague.

\end{appendices}

\openright
\end{document}

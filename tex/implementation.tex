\chapter{Implementation}
\label{chap:implementation}

I have implemented the map-merging algorithm (chapter~\ref{chap:mergingalgorithm}) in the \gls{ROS} framework (section~\ref{sec:ros}). This allows the implementation to be used with various readily available \gls{SLAM} algorithms and inter-robot communication solutions.

Pointcloud algorithms are implemented with \gls{PCL} (section~\ref{sec:pcl}). \gls{PCL} is a popular open-source library for manipulating point-cloud data.

\section{Robot Operating System}
\label{sec:ros}

\gls{ROS} is a popular open-source robotics middleware and distribution of community-maintained robotics software. \gls{ROS} provides hardware abstraction, low-level sensors control, implementations of commonly-used robotics algorithms, including several variants of \gls{SLAM}, message-passing between processes, build system, and package management.

Core of the \gls{ROS} is a messaging system, that allows loose coupling between sensors, robotic algorithms and actuators. There are many community-established standard message formats for various sensor types, visualisation, maps, robot state, linear algebra etc., which promote software reuse.

Software in the \gls{ROS} project is organised in a form of community-maintained packages. First type of packages integrates particular hardware sensors, actuators or entire robot platforms with \gls{ROS} and provide a message interface using standard \gls{ROS} messages. Second type of packages implement robotics algorithms, that usually depend only on standard \gls{ROS} messages, making them reusable between robots. The \gls{ROS} project builds and distributes binary packages for Linux.

\section{Point Cloud Library}
\label{sec:pcl}

\gls{PCL} is an open-source library for pointcloud processing. It contains routines for feature estimation, surface reconstruction, registration, nearest neighbour search, model fitting, including \gls{RANSAC} algorithm, and segmentation. The library offers serialisation of pointclouds to its native, but widely supported \texttt{pcd} file format. \Gls{PCL} is supported in \gls{ROS}, with \gls{PCL} pointcloud format being supported for message serialisation into \gls{ROS} messages.

I have used \gls{PCL} to implement the presented map-merging algorithm (chapter~\ref{chap:mergingalgorithm}) and to read and save \texttt{pcd} files for the command-line tools (section~\ref{sec:commandline-tools}).

\section{\texttt{map\_merge\_3d} package}
\label{sec:ros-package}

The software for this thesis is organised in a \gls{ROS} package \texttt{map\_merge\_3d}. This package contains \gls{ROS} node for online map-merging as well as command-line applications for offline map-merging and visualisation of pointcloud files.

The \texttt{map\_merge\_3d} package is distributed within \gls{ROS} starting from \gls{ROS} Melodic release. Package documentation is maintained as wiki text available online, a reproduction of the documentation is attached (attachement~\ref{chap:map_merge-wiki}).

Care has been taken to integrate the package with the rest of the \gls{ROS} ecosystem. The package is not tied to any particular package for inter-robot communication (section~\ref{sec:communication}) and it uses commonly used map representation in \gls{ROS} (section~\ref{sec:map-representation}).

The \gls{ROS} node supports auto-discovery of the robots (section~\ref{sec:configuration}) and is designed to partially mitigate the high computing demands of the map-merging (section~\ref{sec:node-architecture}). The command-line tools included in the package are described in section~\ref{sec:commandline-tools}.

\subsection{Communication}
\label{sec:communication}

The \gls{ROS} node is designed work with any software for inter-robot communication. This architecture allows user to use any kind of communication media and software solution that is available for a given multi-robot system. The package therefore does not provide nor expect any particular communication between robots.

To achieve loose coupling with communication software, the node expects map topics to be available within local \gls{ROS} graph. User of the package is then expected to setup communication between robots that publishes robot maps within local \gls{ROS} graph. There are several solutions available within \gls{ROS} to achieve this setup.

First of all, \gls{ROS} natively supports sending messages between multiple computers connected in the same network. However the \gls{ROS} master directory listing (broker) service is not distributed and it needs to run on one of these computers. This creates a single point-of-failure that is not acceptable for unreliable communication media.

To achieve reliable multi-robot communication \gls{ROS} Multimaster Special Interest Group was formed with intent to support running multiple \gls{ROS} master broker services. \citet{hernadez2015multi} developed \texttt{multimaster\_fkie} package, that deals with most of the problems of the multi-robot communication. This package can be used together with \texttt{map\_merge\_3d} to run online map-merging for multiple robots.

Users are free to choose any other communication solution meeting their needs for the particular application, \texttt{map\_merge\_3d} does not depend on any particular multi-master setup and can run also under single-master setup, for example in simulated multi-robot environment.

\subsection{Map representation}
\label{sec:map-representation}

Choosing a correct map representation is important to allow interoperability and re-usability within the \gls{ROS} ecosystem. We need to select a map format that is supported by the most \gls{SLAM} implementations in \gls{ROS} to be able to consume maps from the robots.

Widespread \gls{3D} \gls{SLAM} approaches in \gls{ROS} produce maps as \texttt{sen\-sor\_msgs/\-Point\-Cloud2} messages, which are now de-facto standard for representing \gls{3D} maps in \gls{ROS}. This representation is suitable for the presented map-merging algorithm (chapter~\ref{chap:mergingalgorithm}).

Unfortunately there is no standard message format in \gls{ROS} to represent maps as a pose graph of pointclouds, which would allow to implement map-merging on pointcloud pose graph discussed in section~\ref{sec:map-merging-on-pointclouds}, which allows to repair mapping errors.

The map-merging node therefore uses monolithic maps represented in \texttt{sen\-sor\_msgs/\-Point\-Cloud2} messages, which always encode whole map, to allow the highest degree of interoperability with existing \gls{ROS} packages.

Apart from not being able to correct mapping errors, a monolithic map representation also causes unnecessary transfers of already-explored space, that slows map update rate for large-scale maps. A standardised pose graph map representation in \gls{ROS} might help to solve both issues.

% mention rtabmap, google cartographer?

\subsection{Configuration}
\label{sec:configuration}

The \gls{ROS} node supports merging maps from an arbitrary number of robots. To make configuration of the node easy, maps for map-merging are auto-discovered by the \gls{ROS} node.

The node periodically scans the \gls{ROS} graph to discover map topics, which are then subscribed and passed to the map-merging procedure. This mechanism also enables robots to be added to the multi-robot system later, for example to tackle a high demand for the service.

Full documentation of this mechanism is described in attachment~\ref{chap:map_merge-wiki}.

\subsection{ROS node architecture}
\label{sec:node-architecture}

Estimating transformations between maps using the map-merging algorithm (chapter~\ref{chap:mergingalgorithm}) is computationally intensive and can take a long time, especially when merging many large-scale maps.

To achieve high update rates of the merged map, the \gls{ROS} node uses an asynchronous architecture. Because the map origins does not change, we don't need to re-estimate transformations between maps every time we are updating the merged map, we can use the previous estimates to update the map.

The node runs asynchronously the map composition and the transformations estimation tasks, both of the tasks run periodically with user-settable frequencies. The map composition uses already estimated transformations between maps, transforms the input maps and concatenates the maps to produce the merged map. This process is fast and allows the node to achieve high update rates of the merged map to quickly incorporate newly-discovered areas by the robots.

The transformations estimation task uses computationally demanding algorithm (chapter~\ref{chap:mergingalgorithm}) to find the transformations between maps with enough overlapping space. This task can run with lower frequency to save computational resources.

\subsection{Offline map-merging and visualisation}
\label{sec:commandline-tools}

The \gls{ROS} package contains also two command-line applications beside the main \gls{ROS} node. The applications works with pointclouds saved in \texttt{pcd} files.

\texttt{map\_merge\_tool} can be used for offline map-merging. It accepts $n$ pointcloud files and produces a single merged map saved to a file. It uses the same algorithm for map-merging as the \gls{ROS} node. Typical usage might include multi-session mapping of large environment with a single robot and producing a complete map for later robot deployment.

The second tool, \texttt{registration\_visualisation}, accepts two pointclouds and visualises pair-wise transformation estimation algorithm as described in chapter~\ref{chap:mergingalgorithm}. The tool is mostly controlled by the command-line arguments, but the visualisation window is graphical. Each step of the estimation is visualised in \gls{3D}, user is able to freely navigate the pointclouds or save the visualisation as an image. This application can be used for estimation parameter tuning and learning about the map-merging process.

% todo screenshot

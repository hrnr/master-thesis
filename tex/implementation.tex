\chapter{Implementation}
\label{chap:implementation}

I have implemented the map-merging algorithm (chapter~\ref{chap:mergingalgorithm}) in the \gls{ROS} framework (section~\ref{sec:ros}). This allows the implementation to be used with various readily available \gls{SLAM} algorithms and inter-robot communications.

Pointcloud algorithms are implemented with \gls{PCL} (section~\ref{sec:pcl}). \gls{PCL} is a popular open-source library for manipulating point-cloud data.

\section{Robot Operating System}
\label{sec:ros}

\gls{ROS} is a popular open-source robotics middleware and distribution of community-maintained robotics software. \gls{ROS} provides hardware abstraction, low-level sensors control, implementations of commonly-used robotics algorithms, including several variants of \gls{SLAM}, message-passing between processes, and package and build management.

Core of the \gls{ROS} is a messaging system, that allows loose coupling between sensors, robotic algorithms and actuators. There are many community-established standard message formats for various sensor types, visualisation, maps, robot state, linear algebra etc., which allows software reuse.

Software in the \gls{ROS} project is organised in the form of community-maintained packages. First type of packages integrate particular hardware sensors, actuators or entire robot platforms with \gls{ROS} and provide a message interface using standard \gls{ROS} messages. Second type of packages implement robotics algorithms, that usually depend only on standard \gls{ROS} messages, making them reusable between robots. The \gls{ROS} project builds and distributes binary packages for Linux.

\section{\texttt{map\_merge\_3d} package}

The software for this thesis is organised in a \gls{ROS} package \texttt{map\_merge\_3d}. This package contains \gls{ROS} node for online map-merging as well as command-line applications for offline map-merging of pointcloud files.

The \texttt{map\_merge\_3d} package is distributed within \gls{ROS} starting from \gls{ROS} Melodic release. Package documentation is maintained as wiki text available online, reproduction of the documentation is attached (attachement~\ref{chap:map_merge-wiki}).

\subsection{Communication}

The \gls{ROS} node is designed work with any communication solution between robots. This allows user to use any kind of communication media and software solution that is available for a given application. The package therefore does not provide nor expect any particular communication between robots.

To achieve loose coupling with communication setup, the node expects map topics to be available within local \gls{ROS} graph. User of the package is then expected to setup communication between robots that publishes robot maps within local \gls{ROS} graph.

\gls{ROS} natively supports sending messages between multiple computers connected in the same network. However as the \gls{ROS} master directory listing (broker) service is not distributed, it needs to run on one of these computers. This single point-of-failure is not acceptable for unreliable communication media.

To achieve reliable multi-robot communication \gls{ROS} Multimaster Special Interest Group was formed with intent to support running multiple \gls{ROS} master broker services. \citet{hernadez2015multi} developed a \texttt{multimaster\_fkie} package, that deals with most of the problems of multi-robot setup. This package can be used together with \texttt{map\_merge\_3d} to run online map-merging for multiple robots.

Users are free to choose any other communication solution meeting their needs for the particular application, \texttt{map\_merge\_3d} does not depend on any particular multi-master setup and can run also under single-master setup, for example in simulated multi-robot environment.

\subsection{Map representation}

Choosing a correct map representation is important to allow interoperability and re-usability within the \gls{ROS} ecosystem. We need to select a map format that is supported by the most \gls{SLAM} implementations in \gls{ROS} to be able to consume maps from the robots.

Widespread \gls{3D} \gls{SLAM} approaches in \gls{ROS} produce maps as \texttt{sensor\_msgs/PointCloud2} messages, which is now de-facto standard for representing \gls{3D} maps in \gls{ROS}. This representation is suitable for the presented map-merging algorithm (chapter~\ref{chap:mergingalgorithm}).

Unfortunately there is no standard message format in \gls{ROS} to represent map as a pose graph of pointclouds, which would allow to implement map-merging on pointcloud pose graph discussed in section~\ref{sec:map-merging-on-pointclouds}, which allows to repair mapping errors.

The map-merging node therefore uses maps represented in \texttt{sensor\_msgs/PointCloud2} messages, which always encodes whole map.

% todo diasdvantages, rtabmap, google cartographer

\subsection{Configuration}

% auto discovery

\subsection{Estimation}

% threading model of the node

\subsection{Offline map-merging}

The \gls{ROS} packages contains also two command-line applications beside the main \gls{ROS} node. The applications works with pointclouds saved in \texttt{pcd} files.

\texttt{map\_merge\_tool} can be used for offline map-merging. It accepts $n$ pointcloud files and produces a single merged map saved to a file. It uses the same algorithm for map-merging as the \gls{ROS} node. Typical usage might include multi-session mapping of large environment with a single robot and producing a large map for further robot deployment.

The second tool, \texttt{registration\_visualisation}, accepts two pointclouds and visualises pair-wise transformation estimations as described in chapter~\ref{chap:mergingalgorithm}. Each step of the estimation is visualised in \gls{3D}, user is able to freely navigate the pointclouds or save the visualisation as an image. This application can be used for estimation parameter tuning and learning about the map-merging process.

% todo screenshot

\section{Point Cloud Library}
\label{sec:pcl}

\gls{PCL} is an open-source library for pointcloud processing. It contains routines for